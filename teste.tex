\documentclass{article}
\usepackage[utf8]{inputenc}
\usepackage{amssymb,amsmath,graphicx,indentfirst}

\usepackage[brazil]{babel}
\usepackage{listings}
\usepackage{color}
\usepackage{enumerate}
 
\definecolor{dkgreen}{rgb}{0,0.6,0}
\definecolor{gray}{rgb}{0.5,0.5,0.5}
\definecolor{mauve}{rgb}{0.58,0,0.82}
 
\lstset{
  language=Python,                
  basicstyle=\footnotesize,           
  numbers=left,                   
  numberstyle=\tiny\color{gray},  
  stepnumber=2,                             
  numbersep=5pt,                  
  backgroundcolor=\color{white},    
  showspaces=false,               
  showstringspaces=false,         
  showtabs=false,                 
  frame=single,                   
  rulecolor=\color{black},        
  tabsize=2,                      
  captionpos=b,                   
  breaklines=true,              
  breakatwhitespace=false,        
  title=\lstname,                               
  keywordstyle=\color{blue},          
  commentstyle=\color{dkgreen},       
  stringstyle=\color{mauve},     
}

\title{Lista 4 - MAE0399 - Análise de Dados e Simulação}
\author{George Othon - NUSP 10349978 \\ Guilherme Gomes - NUSP 11810196 \\ Espedito Ferreira Alves - NUSP 10786552}
\date{Julho 2021}
\\[1cm]

\begin{document}

\maketitle

\begin{flushleft}
\textbf{Exercício} 2
\\[0.5cm]
\end{flushleft}

\begin{enumerate}
    \begin{figure}[htb]
        \centering
        \includegraphics[scale=0.35]{Imagens/Exercício 2A.png}
        \label{fig:my_label}
    \end{figure}
    
    \begin{figure}[htb]
        \centering
        \includegraphics[scale=0.35]{Imagens/Exercício 2B.png}
        \label{fig:my_label}
    \end{figure}
    
\end{enumerate}


\newpage
\begin{flushleft}
\textbf{Exercício} 4
\\[0.5cm]
\end{flushleft}
Considere o arquivo de dados Weekly que faz parte do pacote ISLR
do R . Esses dados possuem 1089 porcentagens de retornos semanais durante
21 anos (de 1990 até 2010)

    \begin{enumerate}[a.]
    \item Faça alguns grácos resumos dos dados. Você consegue identificar algum padrão? Explique
    
    \newpage
    %%%Lag1
    \begin{figure}[h]
    
    \center
    \subfigure{\includegraphics[width=5cm]{Imagens/lag1 -down.jpeg}}
    \qquad
    \subfigure{\includegraphics[width=5cm]{Imagens/lag1 -up.jpeg}}
    \caption{Lag1}
    \end{figure}
    
    %%%Lag2
    \begin{figure}[h]
    \center
    \subfigure{\includegraphics[width=5cm]{Imagens/lag2 -down.jpeg}}
    \qquad
    \subfigure{\includegraphics[width=5cm]{Imagens/lag2 -up.jpeg}}
    \caption{Lag2}
    \end{figure}
    
    \newpage
    %%%Lag3
     \begin{figure}[h]
    \center
    \subfigure{\includegraphics[width=5cm]{Imagens/lag3 -down.jpeg}}
    \qquad
    \subfigure{\includegraphics[width=5cm]{Imagens/lag3 -up.jpeg}}
    \caption{Lag3}
    \end{figure}
    
    %%%lag4
     \begin{figure}[h]
    \center
    \subfigure{\includegraphics[width=5cm]{Imagens/lag4 -down.jpeg}}
    \qquad
    \subfigure{\includegraphics[width=5cm]{Imagens/lag4 -up.jpeg}}
    \caption{Lag4}
    \end{figure}
    
    \newpage
    %%%Lag5
     \begin{figure}[h]
    \center
    \subfigure{\includegraphics[width=5cm]{Imagens/lag5 -down.jpeg}}
    \qquad
    \subfigure{\includegraphics[width=5cm]{Imagens/lag5 -up.jpeg}}
    \caption{Lag5}
    \end{figure}
    
     %%%Volume
     \begin{figure}[h]
    \center
    \subfigure{\includegraphics[width=5cm]{Imagens/volume-down.jpeg}}
    \qquad
    \subfigure{\includegraphics[width=5cm]{Imagens/volume-up.jpeg}}
    \caption{volume}
    \end{figure}
     \end{enumerate}
     

Quando comparamos a distribuição dos lags e do volume quando separado por classe (Up e Down) apenas no Lag2 podemos observar uma pequena diferença,o que nos da indícios de que o preditor deve explicar o target um pouco melhor que as outras variáveis, o que é contraintuitivo, já que o Lag2 representa a cotação duas semanas atrás e o Lag1 (cotação da semana anterior) parece explicar menos.\\
    
\newpage
\begin{enumerate}[b.]
    \item  Ajuste um modelo de regressão logística aos dados completos considerando Direction como variável resposta e como preditores as cinco variáveis lag mais volume. Use a função summary para apresentar os resultados.
    Quais preditores são estatisticamente significantes? Por que? Especlique as hipóteses que estão sendo testadas.

    \begin{figure}[htb]
        \centering
        \includegraphics[scale=0.5]{Imagens/4BCodigo.png}
        \caption{Código}
        \label{fig:my_label}
    \end{figure}
    
    \begin{figure}[htb]
        \centering
        \includegraphics[scale=0.5]{Imagens/4Bsummary.png}
        \caption{Summary}
        \label{fig:my_label}
    \end{figure}

\\
Queremos testar a hipótese de que todos os parâmetros são nulos, e ao observar o p-valor, considerando um nível de significância de 5\%, temos o parâmetro $\beta_2$ que acompanha o preditor Lag2 estatisticamente significante com probabilidade $0.02$, logo, rejeitamos a hipótese de que $\beta_2 = 0$, e para os demais $\beta$s não podemos rejeitar a hipótese de que são nulos (Para $\beta_0$ podemos rejeitar a hipóteses de que $\beta_0 = 0$, pois o p-valor é bem menor que 5\%. 


\newpage
    \begin{enumerate}[c.]
        \item Obtenha a matriz de classificação (confusion matrix), a sensibilidade, a especificidade $p_c$ = 0.5.
        Comente os resultados.
        
        \begin{figure}[htb]
        \centering
        \includegraphics[scale=0.5]{Imagens/4Csummary.png}
        \caption{Summary}
        \label{fig:my_label}
        \end{figure}
        
        \begin{figure}[htb]
        \centering
        \includegraphics[scale=0.4]{Imagens/4Cconfusion_matrix.png}
        \caption{Matrix de Confusão}
        \label{fig:my_label}
        \end{figure}
        
    Para o limiar $p_c=0.5$ o modelo tem uma acurácia de 0.56 (fração do total de predições corretas), tem uma especificidade de 0.92, ou seja, o modelo prevê corretamente boa parte das informações 
    
    \end{enumerate}
    
    
    
\newpage

    \begin{enumerate}[d.]
        \item Ajuste agora um modelo de regressão logística considerando apenas um período de treinamento, de 1990 a 2008. Considere como preditor apenas a variável lag2. Use a função summary para apresentar e analisar os resultados. Considerando como amostra de teste os anos seguintes (2009 e 2010), obtenha a matriz de classificação (confusion matrix) , a sensibilidade,a especificidade e a fração total de predições corretas.
        \\[0.6cm]
     \begin{figure}[htb]
        \centering
        \includegraphics[scale=0.5]{Imagens/4DCódigo.png}
        \caption{Código}
        \label{fig:my_label}
        \end{figure}
        
        \begin{figure}[htb]
        \centering
        \includegraphics[scale=0.5]{Imagens/4Dsummary.png}
        \caption{Summary}
        \label{fig:my_label}
        \end{figure}
        
        \begin{figure}[htb]
        \centering
        \includegraphics[scale=0.5]{Imagens/4DConfusion_matrix.png}
        \caption{Matrix de Confusão}
        \label{fig:my_label}
        \end{figure}
        
        Do modelo ajustado considerando \textit{Lag2} como preditor da variável resposta \textit{Direction}, podemos ver que o preditor é relevante para um teste com nível de signifiância de 5\%, pois o $p-valor$ vale 0.0417. Dessa forma, rejeitamos a hipótese nula de que $\beta = 0$. A reta ajustada e as probabilidades preditas são dadas por:
    $$
    logito(p(x_{Lag2}))} = 0.550465 + 0.014232x_{Lag2}
    $$
    $$
    \widehat{p}(x_{Lag2}) = \frac{e^{0.550465 + 0.014232x_{Lag2}}}{1 + e^{0.550465 + 0.014232x_{Lag2}}}
    $$
    
    Outra observação que podemos fazer é que a precisão nas predições aumentou, sendo agora 0.625, contra 0.5629 para o modelo no qual utilizamos mais de um preditor. Além disso, houve um aumento na sensibilidade (verdadeiro positivo), o que justifica a maior proporção de acerto nas predições.
        \end{enumerate}
    
\end{document}